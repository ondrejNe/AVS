\chapter{Motivation}\label{ch:motivation}

The aim of the project is to create a C/C++ application designed as a distributed embedded DAQ (Data Acquisition)
system that would simulate functionalities comparable to those found in a simple experimental rocket.
Such a system would be able to monitor and process various environmental parameters, such as pressure or
acceleration, and would on them provide real-time feedback to either itself or the operator.

However, due to the complex nature, extensiveness and the limited scope of the project, the proposed design focuses on
to be an exemplary model of the real-world application.
As the primary goal is, to illustrate the capabilities, only the basic concepts as distributed
telemetry acquisition and transfer will be explored.
Also, the emphasis is given on builtin visualization for the verification of the correct functionality.

To iterate, the main features of the proposed project are the data acquisition from various sources.
Distributed retrieval and transfer of these sources across multiple processing units, and later visualization in a
user-friendly manner, showcasing the system's ability to provide comprehensive real-time analysis of the surrounding
environment.

\section{Requirements}\label{sec:system-requirements}
The proposed system fulfills the following minimal requirements:
\begin{itemize}[leftmargin=*] % Remove the bullet
    \item \textbf{Data Acquisition:} Telemetry collection from multiple peripheral devices.
    \item \textbf{Distributed processing:} Multiple units connected to a single communication network.
    \item \textbf{Data Visualization:} Graphical representation of the acquired data in real-time.
\end{itemize}

\newpage

\section{Project scope}\label{sec:scope}
The project core revolves around a main MCU (Microcontroller Unit) responsible for terminal data aggregation and
visualization.
Multiple secondary MCUs responsible for peripheral data acquisition, CAN (Controller Area Network) bus representing
joint communication medium, underlying control software and final project documentation.

The pivotal role plays the acquired telemetry, which illustrates various environmental parameters.
However, close to none additional processing is done on the acquired data as the primary focus will be on the handling
of the data throughout the system and not its utilization.
For the purpose of this project, the following data representatives are considered:
\begin{table}[ht]
    \centering
    \begin{tabular}{lll}
        \toprule
        \textbf{Data} & \textbf{Description} & \textbf{Source} \\
        \midrule
        Pressure & Measured in Pa & BMP \\
        Temperature & Measured in °C & BMP \\
        \midrule
        Acceleration & Measured in m/s\(^2\) in G & 9-DOF IMU \\
        Rotation & Measured in degrees/s & 9-DOF IMU \\
        Magnetic field & Measured in microteslas (\(\mu\)T) & 9-DOF IMU \\
        \midrule
        Digital Inputs & Binary state & GPIO \\
        \bottomrule
    \end{tabular}
    \caption{Acquired telemetry data}
    \label{tab:telemetry_data}
\end{table}

Terminal data aggregation is displayed to the end user/operator in a concise form.
To correctly verify the system's functionality, the data should be visualized in a user-friendly manner.
For this purpose, the main MCU has at its disposal multiple graphical peripherals.
To name a few, an LCD (Liquid Crystal Display), 7-segment display, alphanumeric display and multiple LED
(Light Emitting Diode) indicators are available.
The built-in button and endpoint inputs also allow for the user to interact with the system and influence the
final graphical output.

Software-wise, the project is dived into \textit{layers}, \textit{modules} and \textit{libraries}.
\textit{Layers} represent the separation of concerns, where each layer is responsible for a specific task or function.
On the other hand, \textit{modules} represent the physical separation of the codebase for a specific MCU\@.
\textit{Libraries} are shared among the modules and layers.
They provide the necessary functionality and allow for control with the underlying hardware.
